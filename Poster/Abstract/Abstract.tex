\documentclass[11pt]{scrartcl}
\usepackage[latin1]{inputenc}
\usepackage[german]{babel}
\usepackage[T1]{fontenc}
\usepackage{latexsym}
\usepackage{stmaryrd}
\usepackage{amsmath}
\usepackage{amssymb}
\usepackage{amsxtra}
\usepackage[round]{natbib}
\usepackage{listings}  
\usepackage{graphicx}  

\renewcaptionname{german}{\contentsname}{Inhalt}
\renewcaptionname{german}{\listfigurename}{Abbildungen}
\renewcaptionname{german}{\listtablename}{Tabellen}
\renewcaptionname{german}{\figurename}{Abb.}
\renewcaptionname{german}{\tablename}{Tab.}
\newcaptionname{german}{\lstlistingname}{Algorithmus}

\newtheorem{theorem}{Theorem} 
\newtheorem{definition}[theorem]{Definition} 


\begin{document}

\section*{{\large Predicting I/O-performance in HPC using Artificial Neural Networks}}
% of the most efficient file access parameters and patterns can be a crucial factor to program performance while being difficult to find.

The prediction of file access times is an important part for the modelling of the storage systems of super computers.
These models can then be used to develop analysis tools which support the integration of efficient I/O-behaviour.\\
We analysed the parallel storage system of a super computer by measuring file access times in various test series.
Afterwards different models were developed and tested in their ability of predicting access times.
Thereby models utilizing artificial neural networks achieved better results than linear regression models.\\
A phenomenon in the measurements of file accesses stands out in particular: File accesses with equal parameter values arranged in multiple groups that are differentiated with a step in the magnitude of access times.\\
These steps in the access times can be explained with a different processing of the file accesses in the storage system, where different I/O-paths have been used.
Because I/O-paths are unknown, knowledge about them has to be derived from other factors.
On the one hand we tried to exploit time dependencies occurring in the measurements, this however did not lead to significant indications about I/O-paths, on the other hand a method was used, that utilizes residues of simple models like linear regression.
To obtain an approximation of I/O-paths the residues were clustered into classes that correspond to I/O-paths.
The analysis of these classes showed that they contain meaningful information about I/O-paths that can be used for better access time predictions.

\end{document}

