\documentclass{superfri}

\usepackage{amssymb}

% ------------

\bibliographystyle{plain}
\begin{document}

%\classify{MSC?}
\author{I.M.~Scientist\footnote{\label{susu}South Ural State University} \and U.R.~Author\footnoteref{susu}}

\title{Predicting I/O-performance in HPC using Artificial Neural Networks}

\maketitle{}

\begin{abstract} %Zusammenfassung
	
The prediction of file access times is an important part for the modeling of the storage systems of super computers. These models can be used to develop analysis tools which support the integration of efficient I/O-behavior.\\
We analyzed the parallel storage system of a super computer by measuring file access times in various test series. Afterwards, different models were developed and tested in their ability of predicting access times. Thereby, models utilizing artificial neural networks achieved better results than linear regression models.
A phenomenon in the measurements of file accesses stands out in particular:
File accesses with equal parameter values have several typical access times.
The steps in the magnitude between these typical access times can be explained with a different processing of the file accesses in the storage system.\\
We developed a method to quantify the significance of knowledge about the internal processing for the prediction of file access times and it proved to be essential.

\keywords{file systems, performance, predicting file access times, artificial neural networks}
\end{abstract}

% -----------------------------------------------------------------------
\section*{Introduction} %Problemdarlegung
\label{sec:intro}

Tools are demanded that help users of HPC-facilities to implement efficient input/output (I/O) in their programs.
It is difficult to find the best access parameters and patterns due to complex parallel storage systems.
Currently users have to optimize their programs at great expense to each system individually without much assistance.
To develop tools which support the implementation of efficient I/O a computational model of the storage system is key.\\
For single hard disk systems such a model can be derived analytically \cite{Ruemmler94anintroduction}; however, for the complex storage system of a super computer these models become too difficult to configure \cite{DBLP:conf/npc/ZhangLZJC10}.
Therefore we searched for good predictors of I/O-performance using a machine learning approach with artificial neural networks (ANNs).\\
In our analysis we used ANNs with different input information for the prediction of access times.
Because of the strong correlation between access time and access size the problem seems to fit linear models.
Our results, however, show that the relation of file access parameters to access time is not sufficiently represented by linear models.
ANNs achieve significantly better results than linear models.\\
The processing of file accesses in a storage system can be viewed as a task that is sequentially propagated along a I/O-path in the storage system.
Starting at the invoking processor the storage system is searching for the data going further and further through the storage hierarchy until all data is found, so it can be returned to the processor.
Our analysis suggests that the I/O-path used by the storage system significantly influences the file access time.
Therefore it becomes key for a good model of access times to derive knowledge about I/O-paths.
Unfortunately I/O-paths are difficult to deal with, as it is unknown which path was used for a file access.

\section{Related work}
\label{sec:related}

Generally storage systems are modeled in two different ways for access time prediction: With white-box- or black-box-modeling \cite{Crume:2013:FML:2538542.2538561}.
\begin{itemize}
	\item \textbf{White-box-modeling}: The storage system itself is simulated. Details of hardware components like rotation speed of the magnetic disk in a hard drive are considered. The processing of a file access can be simulated in the model and the resulting access time is then used as prediction for the actual system.
	Processing and resulting performance can be analyzed in detail on the model.
	\item \textbf{Black-box-modeling}: The model abstracts from the real storage system. 
	System performance is imitated without consideration of its occurrence. %Zustandekommen
	This procedure can also be called emulating.
	In contrast to the white-box-model processing of file accesses can't be analyzed on the model itself.
\end{itemize}
The two ways of modeling are fundamentally different and have to be differentiated.

\subsection{White-box-modeling versus black-box-modeling}
For the in-depth analysis of reasoning for behavior of a storage system a white-box-model is desirable.
On the one hand the modeled system is represented in the model and can thus be examined, on the other hand these models can be very precise if modeled correctly \cite{Ruemmler94anintroduction}.
The problems of white-box-modeling are, however, as obvious as its flaws; they have to be modeled individually for every system and modeling becomes quite intricate for single hard drives \cite{Crume:2013:FML:2538542.2538561}.
To approach the complexity of white-box-modeling Ruemmler and Wilkes analyzed the relevance of different hard drive components for the model deviation to save effort for insignificant parts \cite{Ruemmler94anintroduction}.
However, white-box-modeling is usually used for simple systems like a single hard drive. For these hard drives white-box-modeling is already very demanding, hence for the complex parallel storage system of a super computer it's not a feasible approach \cite{DBLP:conf/npc/ZhangLZJC10}.\\

Application of black-box-modeling is easier and more flexible as it's independent from the individual system.
Stochastic approaches coupled with data mining methods are mostly used for black-box-modeling; for example a combination of regression trees support vector regression \cite{Dai:2012:SDP:2477169.2477214}, or selective bagging classification and regression trees \cite{DBLP:conf/npc/ZhangLZJC10}.

\subsection{Prediction of I/O-performance with ANNs}
Computability of ANNs was reasearched by Rojas \cite{Rojas:1996:NNS:235222} and Cybenko \cite{cybenko:mcss} they demonstrated possibility of modeling non linear systems. Cybenko also proved the \textit{universal approximation theorem} which states that feed-forward networks with sufficient complexity exist that can approximate continuous functions on compact subsets of $\mathbb{R}^n$.
	
\subsection{I/O-performance prediction in HPC}

\section{Structure of the analysis}

\subsection{Benchmark-tests}

\subsection{Error classes}

\subsection{Models}

\section{Evaluation}

\subsection{Test system}

\subsection{Analysis of measurements}
\label{sec:measurements}

\subsection{Analysis of error classes}

\subsection{Prediction of file accesses}

\section{Conclusion and future work}


\ack{Put the acknowledgements after the last section, like this.}
\openaccess

\bibliography{literatur}

%\received{September 25, 2013}

\end{document}
